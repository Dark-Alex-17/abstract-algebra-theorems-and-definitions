\section{Automorphisms}

\begin{definition}[Automorphism]
	An isomorphism from a group $G$ onto itself is called an \textit{automorphisms} of $G$.
\end{definition}

\begin{definition}[Inner Automorphism Induced by $\mathbf{a}$]
	Let $G$ be a group, and let $a \in G$. The function $\phi_a$ defined by $\phi_a(x) = axa^{-1}$ for all $x$ in $G$ is called the \textit{inner automorphism of $G$ induced by $a$}.
\end{definition}

\begin{theorem}[Aut($G$) and Inn($G$) Are Groups]
	The set of automorphisms of a group and the set of inner automorphisms of a group are both groups under the operation of function composition.

	When $G$ is a group, we use Aut($G$) to denote the set of all automorphisms of $G$ and Inn($G$) to denote the set of all inner automorphisms of $G$.
\end{theorem}

\begin{theorem}[Aut$\mathbf{(\Z_n) \approx U(n)}$]
	For every positive integer $n$, Aut($\Z_n$) is isomorphic to $U(n)$.
\end{theorem}
