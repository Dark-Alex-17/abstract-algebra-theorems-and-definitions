\section{Classification of Subgroups of Cyclic Groups}

\begin{theorem}[Fundamental Theorem of Cyclic Groups]
	Every subgroup of a cyclic group is cyclic. Moreover, if $\abs{\lr{a}} = n$, then the order of any subgroup of $\lr{a}$ is a divisor of $n$; and, for each, positive divisor $k$ of $n$, the group $\lr{a}$ has exactly one subgroup of order $k$ -- namely, $\lr{a^{n/k}}$.
\end{theorem}

\begin{corollary}[Subgroups of $\mathbf{\Z_n}$]
	For each positive divisor $k$ of $n$, the set $\lr{n/k}$ is the unique subgroup of $\Z_n$ of order $k$; moreover, these are the only subgroups of $\Z_n$.
\end{corollary}

\begin{theorem}[Number of Elements of Each Order in a Cyclic Group]
	If $d$ is a positive divisor of $n$, the number of elements of order $d$ in a cyclic group of order $n$ is $\phi(d)$.
\end{theorem}

\begin{corollary}[Number of Elements of Order $\mathbf{d}$ in a Finite Group]
	In a finite group, the number of elements of order $d$ is a multiple of $\phi(d)$.
\end{corollary}
