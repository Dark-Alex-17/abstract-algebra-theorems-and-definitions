\section{Subgroup Tests}

\begin{theorem}[One-Step Subgroup Test]
	Let $G$ be a group and $H$ a nonempty subset of $G$. If $ab^{-1}$ is in $H$ whenever $a$ and $b$ are in $H$, then $H$ is a subgroup of $G$. (In additive notation, if $a - b$ is in $H$ whenever $a$ and $b$ are in $H$, then $H$ is a subgroup of $G$.)
\end{theorem}

\begin{theorem}[Two-Step Subgroup Test]
	Let $G$ be a group and let $H$ be a nonempty subset of $G$. If $ab$ is in $H$ whenever $a$ and $b$ are in $H$ ($H$ is closed under the operation), and $a^{-1}$ is in $H$ whenever $a$ is in $H$ ($H$ is closed under taking inverses), then $H$ is a subgroup of $G$.
\end{theorem}

\begin{theorem}[Finite Subgroup Test]
	Let $H$ be a nonempty finite subset of a group $G$. If $H$ is closed under the operation of $G$, then $H$ is a subgroup of $G$.
\end{theorem}

\begin{theorem}[$\mathbf{\lr{a}}$ Is a Subgroup]
	Let $G$ be a group, and let $a$ be any element of $G$. Then, $\lr{a}$ is a subgroup of $G$.
\end{theorem}

\begin{definition}[Center of a Group]
	The \textit{center}, $Z(G)$, of a group $G$ is the subset of elements in $G$ that commute with every element of $G$. In symbols,
	\[ Z(G) = \{a \in G\ \vert\ ax = xa,\ \forall\ x \in G\} \]
	[The notation $Z(G)$ comes from the fact that the German word for center is \textit{Zentrum}. The term was coined by J.A. de Séguier in 1904.]
\end{definition}

\begin{theorem}[Center Is a Subgroup]
	The center of a group $G$ is a subgroup of $G$.
\end{theorem}

\begin{definition}[Centralizer of $\mathbf{a}$ in $\mathbf{G}$]
	Let $a$ be a fixed element of a group $G$. The \textit{centralizer of $a$ in $G$}, $C(a)$, is the set of all elements in $G$ that commute with $a$. In symbols,
	\[ C(a) = \{g \in G\ \vert\ ga = ag\} \]
\end{definition}

\begin{theorem}[$\mathbf{C(a)}$ Is a Subgroup]
	For each $a$ in a group $G$, the centralizer of $a$ is a subgroup of $G$.
\end{theorem}
