\section{Definition and Examples of Groups}

\begin{definition}[Binary Operation]
	Let $G$ be a set. A \textit{binary operation} on $G$ is a function that assigns each ordered pair of elements of $G$ an element of $G$.
\end{definition}

\begin{definition}[Group]
	Let $G$ be a set together with a binary operation (usually called multiplication) that assigns to each ordered pair $(a, b)$ of elements of $G$ an element in $G$ denoted by $ab$. We say $G$ is a \textit{group} under this operation if the following three properties are satisfied.
	\begin{enumerate}
		\item \textit{Associativity}. The operation is associative; that is, $(ab)c = a(bc)$ for all $a,b,c$ in $G$.
		\item \textit{Identity}. There is an element $e$ (called the \textit{identity}) in $G$ such that $ae = ea = a$ for all $a$ in $G$.
		\item \textit{Inverses}. For each element $a$ in $G$, there is an element $b$ in $G$ (called an \textit{inverse} of $a$) such that $ab = ba = e$.
	\end{enumerate}
\end{definition}
