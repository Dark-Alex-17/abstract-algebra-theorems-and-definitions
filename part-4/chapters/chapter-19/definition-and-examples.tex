\section{Definition and Examples}

\begin{definition}[Vector Space]
	A set $V$ is said to be a \textit{vector space} over a field $\F$ if $V$ is an Abelian group under addition (denoted by $+$) and, if for each $a \in \F$ and $v \in V$, there is an element $av \in V$ such that the following conditions hold for all $a,b \in \F$ and all $u,v \in V$.
	\begin{enumerate}
		\item $a(v + u) = av + au$
		\item $(a + b)v = av + bv$
		\item $a(bv)=(ab)v$
		\item $1v=v$
	\end{enumerate}
\end{definition}

\begin{remark}
	The members of a vector space are called \textit{vectors}. The members of the field are called \textit{scalars}. The operation that combines a scalar $a$ and a vector $v$ to form the vector $av$ is called \textit{scalar multiplication}. In general, we will denote vectors by letters from the end of the alphabet, such as $u,v,w$, and scalars by letters from the beginning of the alphabet, such as $a,b,c$.
\end{remark}
