\section{Finite Extensions}

\begin{definition}[Degree of an Extension]
	Let $\E$ be an extension field of a field $\F$. We say that $\E$ \textit{has degree $n$ over $\F$} and write $[\E:\F]=n$ if $\E$ has dimension $n$ as a vector space over $\F$. If $[\E:\F]$ is finite, $\E$ is called a \textit{finite extension} of $\F$; otherwise, we say that $\E$ is an \textit{infinite extension} of $\F$.
\end{definition}

\begin{theorem}[Finite Implies Algebraic]
	If $\E$ is a finite extension of $\F$, then $\E$ is an algebraic extension of $\F$.
\end{theorem}

\begin{theorem}[$\mathbf{[\K:\F] = [\K:\E][\E:\F]}$]
	Let $\K$ be a finite extension field of the field $\E$ and let $\E$ be a finite extension field of the field $\F$. Then $\K$ is a finite extension field of $\F$ and $[\K:\F] = [\K:\E][\E:\F]$.
\end{theorem}

\begin{theorem}[Primitive Element Theorem (Steinitz, 1910)]
	If $\F$ is a field of characteristic 0, and $a$ and $b$ are algebraic over $\F$, then there is an element $c$ in $\F(a,b)$ such that $\F(a,b) = \F(c)$.
\end{theorem}
