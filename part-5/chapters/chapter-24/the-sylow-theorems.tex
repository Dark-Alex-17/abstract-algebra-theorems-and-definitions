\section{The Sylow Theorems}

\begin{theorem}[Existence of Subgroups of Prime-Power Order (Sylow's First Theorem, 1872)]
	Let $G$ be a finite group and let $p$ be a prime. If $p^k$ divides $\abs{G}$, then $G$ has at least one subgroup of order $p^k$.
\end{theorem}

\begin{definition}[Sylow $\mathbf{p}$-Subgroup]
	Let $G$ be a finite group and let $p$ be a prime. If $p^k$ divides $\abs{G}$ and $p^{k+1}$ does not divide $\abs{G}$, then any subgroup of $G$ of order $p^k$ is called a \textit{Sylow $p$-subgroup of $G$}.
\end{definition}

\begin{corollary}[Cauchy's Theorem]
	Let $G$ be a finite group and let $p$ be a prime that divides the order of $G$. Then $G$ has an element of order $p$.
\end{corollary}

\begin{definition}[Conjugate Subgroups]
	Let $H$ and $K$ be subgroups of a group $G$. We say that $H$ and $K$ are \textit{conjugate} in $G$ if there is an element in $G$ such that $H = gKg^{-1}$.
\end{definition}

\begin{theorem}[Sylow's Second Theorem]
	If $H$ is a subgroup of a finite group $G$ and $\abs{H}$ is a power of a prime $p$, then $H$ is contained in some Sylow $p$-subgroup of $G$.
\end{theorem}

\begin{theorem}[Sylow's Third Theorem]
	Let $p$ be a prime and let $G$ be a group of order $p^km$, where $p$ does not divide $m$. Then the number $n$ of Sylow $p$-subgroups of $G$ is equal to 1 modulo $p$ and divides $m$. Furthermore, any two Sylow $p$-subgroups of $G$ are conjugate.
\end{theorem}

\begin{corollary}[A Unique Sylow $\mathbf{p}$-Subgroup Is Normal]
	A Sylow $p$-subgroup of a finite group $G$ is a normal subgroup of $G$ if and only if it is the only Sylow $p$-subgroup of $G$.
\end{corollary}
